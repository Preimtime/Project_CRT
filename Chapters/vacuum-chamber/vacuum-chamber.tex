% !TeX encoding = UTF-8
% !TeX spellcheck = en_US
% !TeX root = ../../Thesis.tex

\chapter{Vacuum test chamber}
\label{chap:Vacuum chamber}

\textcolor{red}{ignore from here} \\
Vacuum chamber -> Frank \\
Kurz: Wie sieht die Kammer aus, ev. Wie ist die CRT drinnen gemountet \\
CRT Mount ??


2020-08-30 leak rate
2020-09-27 set voltages
2020-09-30 first successful external run
2020-10-07 spot vs pressure
2020-10-22 current measurement aluminum foil
2020-11-05 forgot to turn off filament heating
2020-11-14 assemble chamber with copper rings \\
\textcolor{red}{to here}

In order to be able to fit the CRT screen, CF160 flanges were chosen for the test chamber. At one point during testing major changes were made, resulting in a differentiation by iteration.

\section{First iteration}
\label{sec:vacuum chamber first iteration}

A 3D render of the chamber is shown in  (\cref{fig:3D rendering of test chamber}).
 
\subsection{Parts}
\label{subsec:Parts}
 
The center piece consists of a 6-way cross with view ports at the front and bottom. A valve was installed at the back in order to flood the chamber with nitrogen\todo{pure nitrogen name?} when installing a new CRT to avoid oxygen poisoning. On the right side, a HiCube 300 Eco turbo pump was installed and on the left side a wobble stick was attached with a wire. A nipple fitting \todo{length} was installed at the top with a 5 port cluster flange, each being of type CF63.
 
In the middle port, a VSH vacuum transducer was installed to measure pressure. This needs a \SI{24}{\volt} dc power supply. On the left, a 19 pin connector \todo{how many pins and model name?} was installed to supply the necessary voltages to the CRT. Two flanges were equipped with four BNC feedthroughs each. One of them was used to connect do the x-, and y-plates, while the other connected to the wobble stick and aluminum foil at the CRT screen. Further explanation will be given in \todo{ref cd:Beam characterization, include picture there}. The last port was capped off by a blank flange.
 
For the inside wires, stranded copper cables were used. The chamber was sealed by rubber gaskets.
 
\begin{figure}[ht]
	\centering
 	
	\includegraphics[width=0.9\textwidth]{./Chapters/vacuum-chamber/test_chamber} % taken from OneNote QuaK/Vacuum Setup/Test vacuum chamber
	
	\caption{3D rendering of test chamber.}
	\label{fig:3D rendering of test chamber}
\end{figure}
 
\subsection{CRT mounting mechanism}
\label{subsec:CRT mounting mechanism}

 Two M8 rods of length \todo{rod length?} were drilled into the cluster flange. On each, a L-piece was installed between two nuts and they were connected by an hose clamp. Two of these were used to secure the CRT inside the nipple facing the cross (\cref{fig:Image of CRT mounting mechanism}).
 

\begin{figure}[h]
	\centering
	
	\missingfigure[figwidth=0.9\textwidth]{Image of CRT mounting mechanism.}
	
	\caption{Image of CRT mounting mechanism.}
	\label{fig:Image of CRT mounting mechanism}
\end{figure}


\subsection{Leak test}
\label{subsec:Leak test}

Before inserting a CRT, a leak test was performed. First, the chamber was set to a pressure of \SI{e-5}{\milli\bar} after which the pump was turned off. The pressure was measured once a minute for a duration \SI{3}{\hour}. The plot is shown in \cref{fig:Leak rate of test chamber after turning off pump}.

\begin{figure}[ht]
	\centering
		
	\begin{tikzpicture}
		% !TeX encoding = UTF-8
% !TeX spellcheck = en_US
% !TeX root = ../../Thesis.tex

\begin{axis}
	[
		%grid=major,
		ymode = log,
		xlabel = time/\si{\minute},
		ylabel = pressure/\si{\milli\bar},
		%yticklabel style={/pgf/number format/sci},
		xmin = 0,
		ymin = 1e-5,
	]
	\addplot[mark=none, black] table [x=t, y=p, col sep=comma]{./Chapters/vacuum-chamber/leak_rate.csv};
\end{axis}
	\end{tikzpicture}
	
	\caption{Leak rate of test chamber after turning off pump.}
	\label{fig:Leak rate of test chamber after turning off pump}
\end{figure}


erwähne verkabelung
gummi vs kupfer dichtung
l-piece
schlauchbinder (hose clamp)